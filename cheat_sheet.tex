\documentclass[a4paper]{article}
\usepackage[utf8]{inputenc}
\usepackage{enumitem}
\usepackage{url}
\usepackage{hyperref}
\hypersetup{
    colorlinks=true,
    linkcolor=blue,
    filecolor=magenta,      
    urlcolor=cyan,
}
\usepackage{caption}
\usepackage{color}

\definecolor{lightgray}{rgb}{.9,.9,.9}
\definecolor{darkgray}{rgb}{.4,.4,.4}
\definecolor{purple}{rgb}{0.65, 0.12, 0.82}

\usepackage{listings}
\lstset{basicstyle=\ttfamily,
  showstringspaces=false,
  commentstyle=\color{red},
  keywordstyle=\color{blue},
  captionpos=b,
  breaklines=true,
  backgroundcolor=\color{lightgray},
  columns=fullflexible,
}
\lstdefinelanguage{javascript}{
  keywords={typeof, new, true, false, catch, function, return, null, catch, switch, var, if, in, while, do, else, case, break},
  keywordstyle=\color{blue}\bfseries,
  ndkeywords={class, export, boolean, throw, implements, import, this},
  ndkeywordstyle=\color{darkgray}\bfseries,
  identifierstyle=\color{black},
  sensitive=false,
  comment=[l]{//},
  morecomment=[s]{/*}{*/},
  commentstyle=\color{purple}\ttfamily,
  stringstyle=\color{red}\ttfamily,
  morestring=[b]',
  morestring=[b]"
}


% *** GRAPHICS RELATED PACKAGES ***
%\usepackage[pdftex]{graphicx}
\usepackage{graphicx}
%\usepackage[dvips]{graphicx}
% to place figures on a fixed position
\usepackage{float}

\usepackage[margin=1in]{geometry}

\title{Blockchains for Industrial IoT - a Tutorial}

% \subtitle{Cheat sheet}

\author{Pal Varga and Ferenc Janky \\ Budapest University of Technology and Economics}


\date{v0.9 \\ 2019}

\begin{document}

\maketitle

\section{Login details}

\begin{itemize}
\item Demo environment: \url{https://blockchain.cnsm2019-tutorial.com/}
\begin{itemize}
\item username: \verb!user<X>! , where is \verb!<X>! is a number,  e.g.: \verb!user11!, the number allocation will happen at the time of starting the hands-on exercises
\item password: same as the username, e.g \verb!user11!
\end{itemize}
\item SSH access details:
\begin{itemize} 
\item host: blockchain.cnsm2019-tutorial.com
\item port: 2222
\item username: tutorial
\item password: cnsm2019
\end{itemize}
\end{itemize}

\section{Attaching to \emph{geth} node}

\begin{itemize}
\item SSH onto host specified above
\item Attach to \emph{geth} blockchain node by issueing command \verb!sudo ./admin/geth_attach!
\item the \emph{tutorial} user's password has to be given again
\end{itemize}

\section{Most frequently used \emph{geth} commands}

\begin{itemize}
\item the \emph{geth} console is a javascript REPL, any valid Javascript code is accepted. The commands below are actually \emph{web3} API calls, for all available commands see \url{https://web3js.readthedocs.io/en/v1.2.1/getting-started.html}

\item \verb!<variable name> = <expression>! assigns the result of expression  \verb!<expression>! to the variable named \verb!<variable>!, e.g. \verb!block34 = eth.getBlock(34)! that means assign the block object of block 34 to the variable named \verb!block34!


\item to define any custom function define a Javascript function and assign it to a variable:

\begin{lstlisting}[language=javascript]
myFunction = function(param1,param2,...) {
 // body of the function
}

// call the function with arguments
myFunction(arg1,arg2,...)
\end{lstlisting}


\item \verb!eth.getBalance(<account>)! for checking the balance of an account, where \verb!<account>! is a string literal having the value of the address of an Ethereum account or a deployed smart contract, e.g. \verb!eth.getBalance("0x9afeD102A10D54Cc6C0E5153752c69B4876A7419")!

\item \verb!web3.toWei(<value>,<dimension>)! for getting the equivalent value in \emph{Wei}s of the specified amount, where \verb!<value>! is a number literal and \verb!<dimension>! is a string literal having the name of a valid etherum metic, e.g. \verb!web3.toWei(3.14,"ether")!

\item \verb!eth.getBlock(<block Number>)!

\item \verb!eth.getTransaction(<block Number>)!

\item \verb!eth.getGasPrice(console.log)!

\item \verb!eth.getTransactionCount(<account>,[<block specifier>])!, get the total transaction count for acount indicated by \verb!<account>!, optionally a block specifier can be supplied, e.g.: \\ \verb!eth.getTransactionCount("0x9afeD102A10D54Cc6C0E5153752c69B4876A7419","pending")! will get the transaction count for the given address including the pending transactions waiting in queue to be mined onto the blockchain

\item \verb!personal.signTransaction(<Transaction object>, <password>)! to sign a transaction with the senders private key represented by \verb!<Transaction object>! object where the sender's account has \verb!<password>! (string literal) as password, where a transaction has the following parameters

\begin{itemize}

\item \verb!from! : the sender's address

\item \verb!to! : the receiver's address

\item \verb!value! : the amount of \emph{Wei}s to be transferred to the receiver from the sender with this transaction

\item \verb!gas! : the amount of gas that can be consumed by the EVM for executing this transaction

\item \verb!gasPrice! : the price to be paid by the sender per units of gas consumed, the maximum price of the transaction will be \verb!value + gas*gasPrice!, if only fraction of the specified gas has been used by EVM, only the proportional price has to be paid while in contrast if the supplied gas was not enough the transaction will be reverted, but the gas price still has to be paid

\item \verb!nonce! : this is different from the block nonce in case of Proof-of-Work. This one indicates the total transaction count of the user in order to prevent double spending

\item \verb!data! : optional field, the additional data for this transaction, the code and data members are stored in this section of the transaction

\item Example:
\begin{lstlisting}[language=javascript]
signedTransaction = personal.signTransaction({
    from: "0x9afeD102A10D54Cc6C0E5153752c69B4876A7419",
    to: "0x3FecF304285303Fba1C34124889Ea1256e9BB0de",
    value: web3.toWei(1,"ether"),
    gas: 200000, 
    gasPrice: 10,
    nonce: eth.getTransactionCount("0x9afeD102A10D54Cc6C0E5153752c69B4876A7419")
    },"user1" )
\end{lstlisting}

\end{itemize}

\item \verb!eth.sendRawTransaction(<Raw Transaction data>)! sends and execute the  provided transaction where \\ \verb!<Raw Transaction data>! is the raw transaction data in a hexadecimal string format, using the signed transaction object the raw transaction can be accessed by the \verb!raw! member variable, e.g.: \\ \verb!eth.sendRawTransaction(signedTransaction.raw)!

\end{itemize}


\section{PowerBid game details}

By playing this game we'll deploy smart contracts and also interact with each other's contract through its well-defined API. This game realizes a sell auction where the subject of the auction is electrical power. There are 2 roles:
\begin{itemize}
\item Consumer , who creates the contract
\item Supplier, who bids on the Consumer's contract
\end{itemize}

The game can be split up to the following main phases (a detailed state diagram is shown on Figure \ref{fig:State-diagram-powerbid}):

\begin{enumerate}
    \item Auction Phase, where the suppliers participate in the sell auction
    \item Consumption Phase, where the consumer consumes the power paid and requested
    \item Withdraw Phase, where the consumer can withdraw his gain (that is the difference between the max. price and the best price) and the supplier can withdraw the funds from the contract
\end{enumerate}


\begin{figure}[H]
    \centering
    \includegraphics[width=0.9\textwidth]{figures/state_diagram.png}
    \caption{The state diagram of the PowerBid smart contract}
    \label{fig:State-diagram-powerbid}
\end{figure}

TODO: describe UI controls to use in each Phase per role. add typical values

TODO: specify game params

TODO: show how the contracts can be evaluated

\section{Smart Contract Programming exercise}

TODO: describe task, provide hints on implementation

\section{Demo environment/source code info}

In case of there's a desire for further experimentation with this blockchain demo environment its Docker image can be pulled from DockerHub and run by 

\begin{lstlisting}[language=bash]
docker pull fecjanky/iot_sc_tutorial:latest
docker run -p 2222:22 -p 80:80 -p 443:443 fecjanky/iot_sc_tutorial:latest
\end{lstlisting}

NB.: port bindings can be changed freely

\end{document}